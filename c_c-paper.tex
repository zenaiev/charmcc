\documentclass[pdftex,twocolumn,epjc3]{svjour3}          % twocolumn
\pdfoutput=1
\RequirePackage[T1]{fontenc}

\RequirePackage{graphicx}
\RequirePackage{mathptmx}      % use Times fonts if available on your TeX system
\RequirePackage{flushend}
\RequirePackage[numbers,sort&compress]{natbib}
\RequirePackage{amsmath}
\RequirePackage[british]{babel} 
\RequirePackage{bm}              
\RequirePackage{lineno}    
\RequirePackage[latin9]{inputenc} 
\RequirePackage{seqsplit}
\RequirePackage{xcolor}
\RequirePackage{textcomp}
\RequirePackage{amssymb}
\RequirePackage{booktabs}
\RequirePackage{xspace}
\journalname{Eur. Phys. J. C}

\usepackage{widetext}
\newcommand{\DIC}{\Delta_{\rm IC}}

\newcommand{\abmp} {ABMP16\xspace}
\newcommand{\nnpdf} {NNPDF3.1\xspace}
\newcommand{\chisq}{\ensuremath{\chi^2}\xspace}
\newcommand{\ndf}{dof\xspace}
\newcommand{\chisqndf}{\ensuremath{\chi^2}/\ndf\xspace}
\newcommand{\xfitter} {\textsc{xFitter}\xspace}
\newcommand{\qcdnum} {\textsc{qcdnum}\xspace}
\newcommand{\lhapdf} {{\textsc{lhapdf}}\xspace}
\newcommand{\xbj}{\ensuremath{x_{\text{Bj}}}\xspace}
\newcommand{\fonll} {{FONLL-B}\xspace}
\newcommand{\ffns} {{FFNS A}\xspace}
\newcommand{\ffnsb} {{FFNS B}\xspace}
\newcommand{\ffthreea} {{HERAPDF2.0 FF3A}\xspace}
\newcommand{\ffthreeb} {{HERAPDF2.0 FF3B}\xspace}

% Hyperref needs to be loaded last
\RequirePackage[colorlinks,citecolor=blue,urlcolor=blue,linkcolor=blue]{hyperref}
\begin{document}
\sloppy

\title{Charge-current paper}

% authors

\author{xFitter Developers' team:
%     Hamed Abdolmaleki       \thanksref{a}
%\and Valerio Bertone         \thanksref{m,b}
%\and Daniel Britzger         \thanksref{dd}
%\and Stefano Camarda         \thanksref{d}
%\and Amanda Cooper-Sarkar    \thanksref{e}
%\and Francesco Giuli         \thanksref{e}
%\and Alexander Glazov        \thanksref{c}
%\and Aleksander Kusina       \thanksref{g}
%\and Agnieszka Luszczak      \thanksref{c,aa}
%\and Fred Olness             \thanksref{h}
%\and Andrey Sapronov         \thanksref{i}
%\and Pavel Shvydkin          \thanksref{i}
%\and Katarzyna Wichmann      \thanksref{c}
%\and Oleksandr Zenaiev       \thanksref{c}
% and Marco Bonvini           \thanksref{l}
}

%\institute{Faculty of Physics, Semnan University, 35131-19111 Semnan,
%  Iran \label{a}
%  \and Department of Physics and Astronomy, VU University, NL-1081 HV
%  Amsterdam, The Netherlands \label{m}
%  \and NIKHEF Theory Group, Science Park 105, 1098 XG Amsterdam, The
%  Netherlands \label{b}
%  \and Physikalisches Institut, Universit{\" a}t Heidelberg, Im Neuenheimer Feld 226, 69120 Heidelberg, Germany \label{dd} 
%  \and CERN, CH-1211 Geneva 23, Switzerland \label{d}
%  \and Particle Physics, Denys Wilkinson Bdg, Keble Road,
%  University of Oxford, OX1 3RH Oxford, UK \label{e}
%  \and Deutsches Elektronen-Synchrotron (DESY), Notkestrasse 85,
%  D-22607 Hamburg, Germany \label{c}
%  \and Institute of Nuclear Physics, Polish Academy of Sciences,
%  ul. Radzikowskiego 152, 31-342 Cracow, Poland \label{g}
%  \and T. Kosciuszko Cracow University of Technology, PL-30-067, Cracow, Poland \label{aa}
%  \and SMU Physics, Box 0175 Dallas, TX 75275-0175, United States of
%  America \label{h}
%  \and Joint Institute for Nuclear Research (JINR), Joliot-Curie 6,
%  141980, Dubna, Moscow Region, Russia \label{i}
%  \and INFN, sezione di Roma 1, Piazzale Aldo Moro 5, 00185 Rome, Italy \label{l}
%}

\date{Received: date / Accepted: date}
% The correct dates will be entered by the editor

\maketitle

\begin{abstract}

  \footnotetext{Preprint numbers: DESY ...\\
    Correspondence: {\tt ...}}
\end{abstract}

%%%%%%%%%%%%%%%%%%%%%%%%%%%%%%%%%%%%%%%%%%%%%%%%%
%\vspace{-1.0cm}
%\begin{flushright}
%DESY Report-XX-XXX\\
%Nikhef/2017-XXX
%\end{flushright}
%%%%%%%%%%%%%%%%%%%%%%%%%%%%%%%%%%%%%%%%%%%%%%%%%
%\begin{figure}[h]
%%\hspace{11.5cm}
%\includegraphics[width=.22\textwidth]{plots/xFitterLogo.pdf}\\
%\end{figure}
%%%%%%%%%%%%%%%%%%%%%%%%%%%%%%%%%%%%%%%%%%%%%%%%%%

\linenumbers

%%%%%%%%%%%%%%%%%%%%%%%%%%%%%%%%
\section{Introduction}

The Deep-inelastic-scattering (DIS) experiments traditionally have
provided important tests of pQCD and are essential to precisely
determine the parton distribution functions (PDFs) at lepton-nucleon
and nucleon-nucleon colliders.
%
In addition to the numerous dedicated fixed-target DIS experiments
that have been performed, HERA used colliding beams of leptons
(electrons and positrons) on a proton beam to investigate the nucleon
structure.
%
The broad kinematic span of the HERA charge-current (CC) and
Neutral-current (NC) DIS data in terms of the negative transverse
momentum squared $Q^2$ and Bjorken variable $x$ ensure that these data
play important role in modern determinations of the parton
distribution functions~\cite{Abdolmaleki:2018jln,Abramowicz:2015mha,Gao:2017yyd}.
 
In the standard model (SM), the charm quark has an important role in the
investigation of the nucleon structure~\cite{Behnke:2015qja,Zenaiev:2016kfl,Abdolmaleki:2017wlg}.
%
In the NC case, the photon--gluon fusion process for charm production was
calculated at ${\mathcal{O}}(\alpha_s^2)$ with the full heavy quark mass dependence
included in the DIS coefficients~\cite{Laenen:1992zk,Laenen:1992xs}.
%
The heavy quark mass
effects in the CC process have been  calculated up to
${\mathcal{O}}(\alpha_s^2)$ in
Refs.~\cite{Gottschalk:1980rv,Gluck:1997sj,Blumlein:2011zu,Alekhin:2014sya}, 
and the recently completed work of Ref.~\cite{Berger:2016inr} provides results 
up to ${\mathcal{O}}(\alpha_s^2)$ at large $Q^2$ for the
$xF_3$ structure function \cite{Behring:2015roa}.


In many of the posited models which extend the SM, the coupling to
``new physics'' is proportional to the particle mass; hence, the heavy
quarks will have an enhanced coupling and provide an optimal testing
ground for these searches.

Heavy quarks also play a critical role in helping us fully
characterize the SM, and the charm quark is especially useful in this
respect as it can provide us direct access to the strange sea quark
distribution.
%
The strange sea has been extensively investigated in a number of
experiments including $W$ boson charm-jet final state which (at LO)
arise from strange--gluon initial states~\cite{Abazov:2014fka,
  Lai:2007dq};
%

Additionally, neutrino/anti-neutrino nucleon DIS charm production has
been studied by a number of experiments including:
%
 CCFR~\cite{Seligman:1997mc},
 NuTev~\cite{Tzanov:2005kr},
 CHORUS~\cite{Onengut:2005kv},
 CDHSW~\cite{Berge:1989hr}
 and
 NOMAD~\cite{Samoylov:2013xoa}.
%
With a sign-selected beam ($\nu/\bar{\nu}$) these experiments can
separately extract the $s(x)$ and $\bar{s}(x)$ distributions. While
the neutrino-DIS experiments provide detailed information on the shape
of the strange distribution, the normalization is a challenge as that
is tied to the beam flux.

Separately, the HERMES collaboration used charged lepton DIS
production of charged kaons to provide a complementary extraction of
$s(x)+ \bar{s}(x)$~\cite{Airapetian:2008qf}.



Additionally,  charm production mediated by electroweak gauge
boson at hadron colliders provide important information on the strange and
charm quark distribution, and this is complementary to the DIS final state charm
quark experiments~\cite{Lai:2007dq}.
%
The Tevatron  measured the charm quark cross
section in association with  a W boson
at CDF~\cite{Aaltonen:2007dm}  and D0~\cite{Abazov:2008qz}, 
but these results were  limited to ~30\% by low statistics.


In lieu of  significant experiential constraints,
many global QCD analyses tie the strange distribution 
to the light sea quarks 
via the relation $s= \bar{s}=r_s[\bar{u}+\bar{d}]/2$;
%
here, the ratio $r_s$ depends on both $x$ and $Q^2$, but
it is often set to a fixed value at the initial $Q^2$ scale~\cite{Kretzer:2003it, Martin:2004ir}.
%
Using inclusive leptonic decays of $W^\pm$ and $Z$ bosons, ATLAS has  measured  $r_s$
at the scale $Q_0 = 1.9$~GeV$^2$ and $x= 0.023$, and obtain a value of~1.19~\cite{Aaboud:2016btc}.

*** THIS PART NOT DONE *******
The LHC tried to provide a more precise
measurement and CMS and ATLAS collaboration performed ... By
eliminated the isoscalar between strange and anti-strange
distribution, the CTEQ \cite{Lai:2007dq} and MSTW \cite{Martin:2009iq}
extracted the strange and anti-strange distribution at NLO.


This paper organized as follow, in the Sec. ....





%%%%%%%%%%%%%%%%%%%%%%%%%%%%%%%%
\section{Theoretical predictions for charm CC production at LHeC}
\label{sec:thpred}

Theoretical predictions are calculated for electroweak charm CC production in $ep$ collisions at the LHeC at centre-of-mass energy $\sqrt{s} = 1.3$ TeV, using a variety 
of heavy flavour schemss. The predictions are provided for unpolarised beams in the kinematic range $100 < Q^2 < 100000$~GeV$^2$, $0.0001 < \xbj < 0.25$. They are calculated as reduced cross sections at different $Q^2$, \xbj and $y$ points.

{\color{blue}
Experimentally, however, not charm quarks but charmed hadrons (or rather their decay products) are registered in the detectors. 
Therefore, extrapolation to the inclusive charm production cross section has to be carried out in a model dependent way.
Furthermore, in CC charm quarks in the final state can be produced via both electroweak and QCD production processes.
The former leads to an odd number of charm quarks in the final state with the W boson having the same electric charge as the sum electric charges of final-state charm quarks, while the latter creates an even number of charm quarks with total electric charge equal to zero.
If the electric charge of the tagged charm quark can be accessed experimentally (e.g.\ when reconstructing D mesons), the QCD contribution can be excluded by taking the difference of the yields in the events with odd and even numbers of charm quarks, otherwise the QCD contribution can be subtracted only in a model dependent way.
}

The charm CC process directly depends on the CKM matrix. Here, the CKM matrix 
elements $V_{cd}$ and $V_{cs}$ are of particular interest. The values used are 
$V_{cd} = 0.2252$ and $V_{cs} = 0.9734$.
Three different heavy-flavour schemes are employed, all including a full 
treatment of charm mass effects up to NLO\footnote{The $O(\alpha_s^2)$ 
corrections quoted earlier are not yet avaialble in the context of the xFitter 
setup.}, i.e. $O(\alpha_s)$, and described here for the particular application 
to charged current electron-proton reactions. 
The standard fixed flavour number scheme (FFNS A) 
uses three light flavours in both PDFs and $\alpha_s$ evolution, while heavy 
flavours (here: charm) are produced exclusively in the matrix element part of 
the calculation. This scheme has been used e.g. for the PDF determinations 
and cross-section predictions of the ABM(P) group~\cite{Alekhin:2017kpj,Alekhin:2018pai}, as well as in the 
FF3A variant of HERAPDF~\cite{Abramowicz:2015mha}, and implemented in 
xFitter through the OPENQCDRAD package~\cite{openqcdrad}. The ``B'' variant of the 
fixed-order-next-to-leading-log scheme (FONLL B) combines the NLO 
$(O \alpha_s)$ massive matrix elements of the fixed flavour scheme with the 
NLO $(O \alpha_s)$ massless treatment of the zero-mass variable flavour number 
scheme (ZM-VFNS), allowing the number of active flavours (3, 4, or 5) to vary 
with scale, and all-order next-to-leading log resummation of (massless) terms 
beyond NLO. 
It thus explicitly includes charm and beauty both in the PDFs and in the 
evolution of the strong coupling constant.
Whenever terms would be double-counted in the merger of the two schemes 
the massless terms are eliminated in favour of retaining the massive terms.  
The FONLL is heavily used by the NNPDF group [Ball:2017nwa] and implemented in 
xFitter through the APFEL package~\cite{Bertone:2013vaa}. 
Finally a variant of the fixed flavour number scheme known as the `mixed' 
scheme or \ffnsb~\cite{Behnke:2015qja} is used. In this scheme, the 
number of active flavours is still fixed (here: to 3) in the PDFs, relying 
exclusively on $O(\alpha_s)$ fully massive matrix elements for charm 
production, while the number of flavours is allowed to vary in the virtual 
corrections of the alphas evolution. 
Corrections to this evolution involving heavy flavour loops are thus
included and resummed to all orders, as in the VFNS schemes, while no 
resummation is applied to other higher order corrections. This procedure 
will catch a large fraction of the ``large logs'' which might spoil the 
fixed-flavour scheme convergence at very high scales, and is possible since the 
masses of the charm and beauty quarks provide natural cutoffs for infrared 
and collinear divergences. This scheme was used in the HERAPDF FF3B variant~\cite{Abramowicz:2015mha} and in applications of the HVQDIS program~\cite{,Behnke:2015qja}. 
In general the transition from the FFNS A to the
FFNS B scheme requires a readjustement of the treatment of matrix elements 
involving heavy flavour loops, but in the specific case of charged current 
no such loops occur up to NLO (at NNLO they will), such that the same 
matrix elements can be used for both schemes, and here the only difference is 
the alphas evolution.
  
In summary, the schemes used are
\begin{itemize}
  \item \ffns with $n_f = 3$ at NLO and \abmp~\cite{Alekhin:2018pai} or \ffthreea~\cite{Abramowicz:2015mha} NLO PDF set,
  \item \ffnsb with $n_f = 3$ at NLO and \abmp~\cite{Alekhin:2018pai} or \ffthreeb~\cite{Abramowicz:2015mha} NLO PDF set,
  \item FONLL-B with $n_f = 3$ and \nnpdf NLO PDF set~\cite{Ball:2017nwa}.
\end{itemize}
All calculations are interfaced in \xfitter and available in the scheme using the running $\overline{MS}$ charm mass, $m_c(m_c)$.
The $\overline{MS}$ charm mass is set to $m_c(m_c) = 1.27$ GeV~\cite{pdg}, and $\alpha_s$ is set to the value used for the corresponding PDF extraction ($\alpha_s(M_Z) = 0.1191$ for \abmp, and $\alpha_s(M_Z) = 0.118$ for \nnpdf).
The renormalisation and factorisation scales are chosen to be $\mu_\mathrm{r} = \mu_\mathrm{f} = Q^2$.

To estimate theoretical uncertainties, the two scales are simultaneously varied up and down by factor $2$. In the case of the \fonll calculations, also the independent $\mu_r$ and $\mu_f$ variations are checked. Furthermore, the PDF uncertainties are propagated to the calculated theoretical predictions, while the uncertainties arising from varying the charm mass $m_c(m_c) = 1.27 \pm 0.03$ GeV are smaller than $1\%$ and therefore neglected. In the \fonll scheme, as a cross check, the calculation was performed with the pole charm mass $m_c^{\text{pole}} = 1.51$ GeV which is consistent with the conditions of the \nnpdf extraction~\cite{Ball:2017nwa}. The obtained theoretical predictions differ from the ones calculated with $m_c(m_c) = 1.27$ GeV by less than $1\%$.
The total theoretical uncertainties are obtained by adding in quadrature the scale and PDF uncertainties.


\subsection{Comparison of theoretical predictions in the \ffns and \fonll schemes}
\label{sec:thpred-comparison}

Figures~\ref{fig:thpred-x}, \ref{fig:thpred-q2} and \ref{fig:thpred-y} show theoretical predictions with their total uncertainties in both schemes as a function of \xbj for different values of $Q^2$, as a function of $Q^2$ for different values of \xbj, and as a function of $y$ for different values of $Q^2$, respectively. The \ffns and \fonll agree reasonably well, within uncertainties of moderate size, in the bulk of the phase space. However, in phase space corners such as high $Q^2 \gtrsim 10000$ GeV$^2$ or low $y \lesssim 0.05$ the predictions in the two schemes differ by more than $50\%$, and these differences are not covered by the theoretical uncertainties.

\begin{figure}
    \centering
    \centering{{\includegraphics[width=0.50\textwidth]{pics/plots-11122018/plot-sigmared-x-em.pdf}}}
    \caption{The theoretical predictions with their total uncertainties for charm CC production at the LHeC as a function of \xbj for different values of $Q^2$ calculated in the \ffns and \fonll schemes. The bottom panel display the theoretical predictions normalised to the nominal values of the \ffns predictions.}
    \label{fig:thpred-x}
\end{figure}

\begin{figure}
    \centering
    \centering{{\includegraphics[width=0.50\textwidth]{pics/plots-11122018/plot-sigmared-q2-em.pdf}}}
    \caption{The theoretical predictions with their total uncertainties for charm CC production at the LHeC as a function of $Q^2$ for different values of \xbj calculated in the \ffns and \fonll schemes. The bottom panel display the theoretical predictions normalised to the nominal values of the \ffns predictions.}
    \label{fig:thpred-q2}
\end{figure}

\begin{figure}
    \centering
    \centering{{\includegraphics[width=0.50\textwidth]{pics/plots-11122018/plot-sigmared-y-em.pdf}}}
    \caption{The theoretical predictions with their total uncertainties for charm CC production at the LHeC as a function of $y$ for different values of $Q^2$ calculated in the \ffns and \fonll schemes. The bottom panel display the theoretical predictions normalised to the nominal values of the \ffns predictions.}
    \label{fig:thpred-y}
\end{figure}

In Fig.~\ref{fig:thpred-q2-unc} the PDF and scale uncertainties of charm CC cross sections as a function of $Q^2$ for different values of \xbj calculated in the \ffns and \fonll schemes are shown. On average, in the \fonll scheme both the PDF and scale uncertainties exceed those in the \ffns scheme. Furthermore, Fig.~\ref{fig:thpred-q2-varmu} shows the impact of separate scale variations in the two schemes. In the \fonll scheme, the variation of $\mu_f$ has a much larger impact on the predictions than the variation of $\mu_r$, and thus it is dominant for the resulting scale uncertainties. {\bf [Valerio, could you please discuss more here?]} Only the simultaneous $\mu_f = \mu_r$ variation is available in the implementation of the \ffns scheme. 
%Overall, the scale variations in the \fonll scheme appears to be larger than those in the \ffns scheme.

\begin{figure*}
    \centering
    \centering{{\includegraphics[width=0.49\textwidth]{pics/plots-11122018/plot-unc-q2-em-FFABM.pdf}}}
    \centering{{\includegraphics[width=0.49\textwidth]{pics/plots-11122018/plot-unc-q2-em-FONLL.pdf}}}
    \caption{Relative theoretical uncertainties of charm CC predictions for the LHeC as a function of $Q^2$ for different values of \xbj calculated in the \ffns and \fonll schemes. The PDF and scale uncertainties are shown separately.}
    \label{fig:thpred-q2-unc}
\end{figure*}

\begin{figure*}
    \centering
    \centering{{\includegraphics[width=0.49\textwidth]{pics/plots-11122018/plot-varmu-q2-em-FFABM.pdf}}}
    \centering{{\includegraphics[width=0.49\textwidth]{pics/plots-11122018/plot-varmu-q2-em-FONLL.pdf}}}
    \caption{The impact of separate scale variations on charm CC predictions for the LHeC as a function of $Q^2$ for different values of \xbj calculated in the \ffns and \fonll schemes.}
    \label{fig:thpred-q2-varmu}
\end{figure*}

To explore whether the differences between the two sets of theoretical predictions appear due to the different treatment of heavy quarks or due to different PDF sets, theoretical calculations in the \ffns and \fonll schemes are repeated with PDF sets extracted from the fit to the HERA DIS data~\cite{Abramowicz:2015mha}. The fit settings follow the HERAPDF2.0 analysis~\cite{Abramowicz:2015mha}. 
In this study, consistent conditions of the PDF extraction eliminate possible differences between the predictions for the LHeC arising from the dissimilarities of the \abmp and \nnpdf analysis. The obtained results are displayed in Figs.~\ref{fig:thpred-fit-y}--\ref{fig:thpred-fit-x}. The differences between the \ffns and \fonll schemes in these predictions are similar to the ones displayed in Figs.~\ref{fig:thpred-x}--\ref{fig:thpred-y} and prove that these differences arise due to the different treatment of heavy quarks in the two schemes.

Furthermore, the predictions in the \ffns and \ffnsb schemes calculated using the \ffthreea and \ffthreeb PDF sets are displayed in Figs.~\ref{fig:thpred-fit-y}--\ref{fig:thpred-fit-x}. Because of the similarities of the two HERAPDF2.0 PDF sets, the differences between the two sets of the predictions arise mainly due to the different treatment of heavy quarks in the two schemes. Remarkably, the differences between \ffns and \fonll predictions are similar to the ones between \ffns and \ffnsb, i.e.\ a larger part of these differences arise due to the different treatment of heavy quarks in $\alpha_s(\mu)$ running.
%{\bf [TODO: Achim, could you please comment more here?]}

{\color{red}
To explore whether the differences between the two sets of theoretical predictions appear due to the different treatment of heavy quarks or due to different PDF sets, theoretical calculations in the \ffns and \fonll schemes are repeated with the HERAPDF2.0 sets extracted from the same HERA DIS data using coherent settings~\cite{Abramowicz:2015mha}. Furthermore, the predictions in the \ffnsb scheme are produced using the \ffthreeb PDF set and the \ffns matrix elements, which are equaivalent to the \ffnsb matrix elemenets at NLO for charm CC production. The obtained results are displayed in Fig.~\ref{fig:thpred-ff3abfonll}. The differences between the \ffns and \fonll schemes in these predictions are similar to the ones displayed in Figs.~\ref{fig:thpred-x}--\ref{fig:thpred-y} and prove that these differences arise due to the different treatment of heavy quarks in the two schemes. The \ffnsb predictions are between the \ffns and \fonll predictions, indicating that a large part of these differences arise due to the different treatment of heavy quarks in $\alpha_s(\mu)$ running.

\begin{figure*}
    \centering
    \centering{{\includegraphics[width=0.49\textwidth]{pics/plots-11122018/plot-FF3ABFONLL-q2-em.pdf}}}
    \centering{{\includegraphics[width=0.49\textwidth]{pics/plots-11122018/plot-FF3ABFONLL-x-em.pdf}}}
    \centering{{\includegraphics[width=0.49\textwidth]{pics/plots-11122018/plot-FF3ABFONLL-y-em.pdf}}}
    \caption{The theoretical predictions for charm CC production at the LHeC as a function of $Q^2$ ($\xbj$, $y$) for different values of $\xbj$ ($\xbj$, $Q^2$) obtained using the HERAPDF2.0 PDF sets in the \ffns, \ffnsb and \fonll schemes. The bottom panel display the theoretical predictions normalised to the nominal values of the \ffns predictions.}
    \label{fig:thpred-ff3abfonll}
\end{figure*}
}

%\begin{figure*}
%    \centering
%    \centering{{\includegraphics[width=0.49\textwidth]{pics/plots-191018/plot-fit-q2-em.pdf}}}
%    \centering{{\includegraphics[width=0.49\textwidth]{pics/plots-191018/plot-FF3AB-q2-em.pdf}}}
%    \caption{(left) The theoretical predictions for charm CC production at the LHeC as a function of $Q^2$ for different values of $\xbj$. See Fig.~\ref{fig:thpred-fit-y} for further details.}
%    \label{fig:thpred-fit-q2}
%\end{figure*}
%
%\begin{figure*}
%    \centering
%    \centering{{\includegraphics[width=0.49\textwidth]{pics/plots-191018/plot-fit-x-em.pdf}}}
%    \centering{{\includegraphics[width=0.49\textwidth]{pics/plots-191018/plot-FF3AB-x-em.pdf}}}
%    \caption{(left) The theoretical predictions for charm CC production at the LHeC as a function of $Q^2$ for different values of $\xbj$. See Fig.~\ref{fig:thpred-fit-y} for further details.}
%    \label{fig:thpred-fit-x}
%\end{figure*}
%
%\begin{figure*}
%    \centering
%    \centering{{\includegraphics[width=0.49\textwidth]{pics/plots-191018/plot-FF3ABFONLL-y-em.pdf}}}
%    \centering{{\includegraphics[width=0.49\textwidth]{pics/plots-191018/plot-FF3ABFONLL-y-em.pdf}}}\\
%    \centering{{\includegraphics[width=0.49\textwidth]{pics/plots-191018/plot-FF3ABFONLL-y-em.pdf}}}
%    \caption{The theoretical predictions for charm CC production at the LHeC as a function of $y$ for different values of $Q^2$ obtained in the fit to the HERA data in the \ffns and \fonll schemes. The bottom panel display the theoretical predictions normalised to the nominal values of the \ffns predictions. (right) Same predictions but obtained in the \ffns and \ffnsb schemes using the \ffthreea and \ffthreeb sets, respectively. The bottom panel display the theoretical predictions normalised to the nominal values of the \ffnsb predictions.}
%    \label{fig:thpred-fit-y}
%\end{figure*}

Furthermore, to investigate the impact of the NNLO corrections available at $Q \gg m_c$ for the FFNS calculation, approximate NNLO predictions are obtained using the \abmp NNLO PDF set~\cite{Alekhin:2017kpj}. The results for the cross sections as a function of $Q^2$ for difference values of \xbj are shown in Fig.~\ref{fig:thpred-q2-nnlo}, where they are compared to the NLO FFNS predictions from Fig.~\ref{fig:thpred-q2}. The NNLO corrections do not exceed $10\%$ and thus do not cover the differences between the \ffns and \fonll theoretical predictions. Similar results are observed for cross sections as functions of other kinematic variables.

\begin{figure}
    \centering
    \centering{{\includegraphics[width=0.50\textwidth]{pics/plots-11122018/plot-FFABMnnlo-q2-em.pdf}}}
    \caption{The theoretical predictions with their total uncertainties for charm CC production at the LHeC as a function of $Q^2$ for different values of \xbj calculated in the \ffns scheme at NLO and approximate NNLO. The bottom panel display the theoretical predictions normalised to the nominal values of the \ffns NLO predictions.}
    \label{fig:thpred-q2-nnlo}
\end{figure}


\color{blue} %%%%%%%%%%%%%%%%%%%%%%%%%%%%%%%%%%%
To better understand the differences between the FFNS  and VFNS calculations, 
Fig.2*** which displays the cross section vs. $Q^2$ is particularly instructive. 
We see at low scales the FFNS and  VFNS results coincide. 
When the $\mu$ scale is below the charm threshold scale (typically taken to be $\sim m_c$)
the charm PDF vanishes and the FFNS and VFNS reduce to the same 
result.\footnote{Note that while the charm threshold scale $\mu_c$ is commonly 
set to the charm quark mass $m_c$, 
the choice of  $\mu_c$ is arbitrary and amounts to a 
renormalzation scheme choice~\cite{Bertone:2017ehk}.
}
%
For increasing scales, the VFNS resums the $\alpha_S \ln(\mu^2/m_c^2)$ contributions
via the DGLAP evolution equations and the FFNS and VFNS will slowly diverge logarithmically.
This behavior is observed in Fig.2*** and consistent with the characteristics demonstrated in 
Ref.~\cite{Kusina:2013slm}. 

More precisely, Ref.~\cite{Kusina:2013slm} used a matched set of $N_F=3$ and $N_F=5$ PDFs
to study the impact of the scheme choice at large scales. They found the resummed contributions 
in the VFNS yielded a larger cross section than the FFNS (the specific magnitude was $x$-dependent), and that for $Q$ scales more than a few times the quark mass, the differences due to scheme choice
exceeded the differences due to (estimated) higher order contributions~\cite{Kusina:2013slm}. 


\color{black} %%%%%%%%%%%%%%%%%%%%%%%%%%%%%%%%%%%



\subsection{Contributions from different partonic subprocesses}
\label{sec:thpred-partonic}

{\bf [perhaps this text would be more appropriate in an earlier theory section]}
The reduced charm CC production cross sections can be expressed as a linear combinations of the structure functions:
\begin{equation}
\begin{split}
    \sigma^{\pm}_{\text{charm,CC}} &= 0.5(Y_{+}F_2^{\pm} \mp Y_{-}xF_3^{\pm} - y^2F_L^{\pm}),\\
    Y_{\pm} &= 1 \pm (1-y)^2.
\end{split}
\end{equation}
In the simplified Quark Parton Model, where gluons are not present, the structure functions become:
\begin{equation}
\begin{split}
    F_2^{+} &= xD + x\overline{U}, \\
    F_2^{-} &= xU + x\overline{D},\\
    F_L &= 0,\\
    xF_3^{+} &= xD - x\overline{U}, \\
    xF_3^{-} &= xU - x\overline{D}.
\end{split}
\end{equation}
The terms $xU$, $xD$, $x\overline{U}$ and $x\overline{D}$ denote the sums of parton distributions for up-type and down-type quarks and anti-quarks, respectively. 
Below the $b$-quark mass threshold, these sums are related to the quark distributions as follows:
\begin{equation}
\begin{split}
 xU &= xu + xc , \\
 x\overline{U} &= x\overline{u} + x\overline{c} , \\
 xD &= xd + xs , \\
 x\overline{D} &= x\overline{d} + x\overline{s}.
\end{split}
\end{equation}
In the FFNS the charm quark density is zero.
In the phase space corners $y \to 0$ and $y \to 1$, the following asymptotics take place:
\begin{equation}
\begin{split}
 y \to 0: \sigma^{\pm}_{\text{charm,CC}} &= F_2^{\pm} = xD(x\overline{D}) + xU(x\overline{U}), \\
 y \to 1: \sigma^{\pm}_{\text{charm,CC}} &= 0.5(F_2^{\pm} \mp xF_3^{\pm}) = xU (x\overline{U}).
\label{eq:y01}
\end{split}
\end{equation}
Thus the contribution from the strange quark PDF is suppressed at high $y$.

Figures~\ref{fig:partonic-x}, \ref{fig:partonic-q2} and \ref{fig:partonic-y} show contributions from different partonic subprocesses for charm CC production cross sections in the \ffns and \fonll schemes as a function of \xbj for different values of $Q^2$, as a function of $Q^2$ for different values of \xbj, and as a function of $y$ for different values of $Q^2$, respectively.
In both scheme, the strange quark PDF contributes only about $50\%$ to total charm CC production. In particular, at high $y$ its contribution drops to zero in favor of the gluon or charm quark PDF (see Fig.~\ref{fig:partonic-y} and Eq.~\ref{eq:y01}). Similar phenomena (although less pronounced) is observed at low \xbj and/or high $Q^2$. In these phase space regions, the dominant contributions to the cross sections are the gluon PDF (in the FFNS) or the charm quark PDF (in the VFNS). Remarkably, these contributions as functions of $Q^2$, \xbj and $y$ behave qualitatively very similar in the FFNS and VFNS.

\begin{figure*}
    \centering
    \centering{{\includegraphics[width=0.49\textwidth]{pics/plots-11122018/plot-parton-x-em-FFABM.pdf}}}
    \centering{{\includegraphics[width=0.49\textwidth]{pics/plots-11122018/plot-parton-x-em-FONLL.pdf}}}
    \caption{The partonic subprocesses for charm CC production cross sections in the \ffns (left) and \fonll (right) schemes as a function of \xbj for different values of $Q^2$.}
    \label{fig:partonic-x}
\end{figure*}

\begin{figure*}
    \centering
    \centering{{\includegraphics[width=0.49\textwidth]{pics/plots-11122018/plot-parton-q2-em-FFABM.pdf}}}
    \centering{{\includegraphics[width=0.49\textwidth]{pics/plots-11122018/plot-parton-q2-em-FONLL.pdf}}}
    \caption{The partonic subprocesses for charm CC production cross sections in the \ffns (left) and \fonll (right) schemes as a function of $Q^2$ for different values of \xbj.}
    \label{fig:partonic-q2}
\end{figure*}

\begin{figure*}
    \centering
    \centering{{\includegraphics[width=0.49\textwidth]{pics/plots-11122018/plot-parton-y-em-FFABM.pdf}}}
    \centering{{\includegraphics[width=0.49\textwidth]{pics/plots-11122018/plot-parton-y-em-FONLL.pdf}}}
    \caption{The partonic subprocesses for charm CC production cross sections in the \ffns (left) and \fonll (right) schemes as a function of $y$ for different values of $Q^2$.}
    \label{fig:partonic-y}
\end{figure*}


\color{blue} %%%%%%%%%%%%%%%%%%%%%%%%%%%%%%%%%%%

Figures 7,8,9*** display a particularly interesting pattern; 
the gluon contribution for the FFNS  is strikingly similar to the charm contribution 
in the VFNS. 

In the FFNS, the charm is produced  predominantly from the explicit process $g \gamma\to c\bar{c}$. 
In contrast, for the VFNS the  $g\to c\bar{c}$ splitting is implicit (internal to the proton and evolved 
with the DGLAP evolution equations); the charm parton then emerges from the proton
to participate in the $c \gamma \to c$ process. 
%
The fundamental underlying process is the same in both the FFNS and VFNS, but 
the factorization boundary between the PDF and the hard scattering cross section, 
$f \otimes \hat{\sigma}$,  
(determined by $\mu$ and the scheme choice) is 
different.\footnote{Note 
there is a ``subtraction'' term  ($g\to c\bar{c} \otimes c \gamma \to c$) which closely 
matches the LO $c \gamma \to c$ process, but this ${\cal O}(\alpha \, \alpha_S)$  process is contained in the NLO gluon-initiated contribution.}



\color{black} %%%%%%%%%%%%%%%%%%%%%%%%%%%%%%%%%%%



\section{PDF constraints from charm CC pseudodata}
\label{sec:PDF}

The impact of charm CC cross section measurements at the LHeC on the PDFs is quantitatively estimated using a profiling technique~\cite{Paukkunen:2014zia}. This technique is based on minimizing \chisq between data and theoretical predictions taken into account both experimental and theoretical uncertainties arising from PDF variations. Two NLO PDF sets were chosen for this study: 
\abmp~\cite{Alekhin:2018pai} and \nnpdf~\cite{Ball:2017nwa} available via the \lhapdf interface (version 6.1.5)~\cite{Buckley:2014ana}. 
All PDF sets are provided with uncertainties in the format of eigenvectors. 

For this study, pseudodata representing measurements of charm CC production cross sections as a function of $Q^2$ and $x$ are used. {\bf [TODO: describe how pseudodata were produced]}
The study is performed using the \xfitter program (version 2.0.0)~\cite{Alekhin:2014irh}, an open-source QCD fit framework for PDF determination. The theoretical predictions are calculated at NLO QCD in the FFNS with the number of active flavours $n_f = 3$ and FONLL-B with $n_f = 5$. The running charm mass is set to $m_c(m_c) = 1.27$ GeV and $\alpha_s$ is set to the value used for the corresponding PDF extraction.
The renormalisation and factorisation scales are chosen to be $\mu_\mathrm{r} = \mu_\mathrm{f} = Q^2$.

The \chisq value is calculated as follows:
\begin{equation}
\chisq = \mathbf{R}^{T}_{} \mathbf{Cov}^{-1}_{} \mathbf{R}_{} + \sum_{\beta} b_{\beta,\rm th}^2,~~~\mathbf{R} = \mathbf{D} - \mathbf{T} - \sum_{\beta} \Gamma^{}_{\beta,\rm th} b_{\beta,\rm th},
\label{eq:chisq}
\end{equation}

where $\mathbf{D}$ and $\mathbf{T}$ are the column vectors of the measured and predicted values, respectively, 
and the correlated theoretical PDF uncertainties are included using the nuisance parameter vector $\boldsymbol{b_{\rm th}}$ with their influence on the theory predictions described by $\Gamma^{}_{\beta,\rm th}$, where index $\beta$ runs over all PDF eigenvectors. 
For each nuisance parameter a penalty term is added to the \chisq, representing the prior knowledge of the parameter. 
No theoretical uncertainties except the PDF uncertainties are considered.
The full covariance matrix representing the statistical and systematic uncertainties of the data is used in the fit. The statistical and systematic uncertainties are treated as additive, i.e., they do not change in the fit. The systematic uncertainties are assumed uncorrelated between bins.

To treat the asymmetric PDF uncertainties of the \nnpdf set, the \chisq function in Eq.~\ref{eq:chisq} is generalised assuming a parabolic dependence of the prediction on the nuisance parameter~\cite{Alekhin:2014irh}:
\begin{eqnarray}
\Gamma^{}_{\beta, \rm th} \to \Gamma^{}_{\beta, \rm th} +  \Omega^{}_{\beta, \rm th}b_{\beta, \rm th}\,, \label{eq:iter}
\end{eqnarray}
where $\Gamma^{}_{\beta, \rm th} = 0.5(\Gamma^{+}_{\beta, \rm th} - \Gamma^{-}_{\beta, \rm th})$ and $\Omega^{}_{\beta} = 0.5(\Gamma^{+}_{\beta, \rm th}
+ \Gamma^{-}_{\beta, \rm th})$ are determined from the shifts of predictions corresponding to up ($\Gamma^{+}_{\beta, \rm th}$) and down ($ \Gamma^{-}_{\beta, \rm th}$) PDF uncertainty eigenvectors.

The values of the nuisance parameters at the minimum, $b^{\rm min}_{\beta,\rm th}$ are interpreted as optimised, or profiled, PDFs, while their uncertainties determined using the tolerance criterion of $\Delta\chi^2 = 1$ correspond to the new PDF uncertainties. The profiling approach assumes that the new data are compatible with theoretical predictions using the existing PDFs, such that no modification of the PDF fitting procedure is needed. Under this assumption, the central values of the measured cross sections are set to the central values of the theoretical predictions. 

The original and profiled \abmp and \nnpdf PDF uncertainties are shown in Figs.~\ref{fig:pdf-abmp}--\ref{fig:pdf-nnpdf-100000}. 
The uncertainties of the PDFs are presented at the scales $\mu_\mathrm{f}^2=100$ GeV$^2$ and $\mu_\mathrm{f}^2=100000$ GeV$^2$.
A strong impact of the charm CC pseudodata on the PDFs is observed for both PDF sets.
In particular, the uncertainties of the strange PDF are strongly reduced once the pseudodata are included in the fit. 
Also the gluon PDF uncertainties are decreased. Furthermore, in the case of the NNPDF3.1 set and FONLL scheme, the charm PDF uncertainties are reduced significantly.

\begin{figure}
    \centering
    {{\includegraphics[width=0.235\textwidth]{pics/pdf-profile-ffabm/q2_100_pdf_s_ratio.pdf}}}
    {{\includegraphics[width=0.235\textwidth]{pics/pdf-profile-ffabm/q2_100_pdf_g_ratio.pdf}}}\\
    {{\includegraphics[width=0.235\textwidth]{pics/pdf-profile-ffabm/q2_100_pdf_Sea_ratio.pdf}}}
    {{\includegraphics[width=0.235\textwidth]{pics/pdf-profile-ffabm/q2_100_pdf_uv_ratio.pdf}}}
    {{\includegraphics[width=0.235\textwidth]{pics/pdf-profile-ffabm/q2_100_pdf_dv_ratio.pdf}}}
    \caption{The relative strange (top left), gluon (top right), sea quark (middle left), u valence quark (middle right) and d valence quark (bottom) PDF uncertainties at $\mu_\mathrm{f}^2=100$ GeV$^2$ of the original and profiled \abmp PDF set.}
    \label{fig:pdf-abmp}
\end{figure}

\begin{figure}
    \centering
    {{\includegraphics[width=0.235\textwidth]{pics/pdf-profile-ffabm/q2_100000_pdf_s_ratio.pdf}}}
    {{\includegraphics[width=0.235\textwidth]{pics/pdf-profile-ffabm/q2_100000_pdf_g_ratio.pdf}}}\\
    {{\includegraphics[width=0.235\textwidth]{pics/pdf-profile-ffabm/q2_100000_pdf_Sea_ratio.pdf}}}
    {{\includegraphics[width=0.235\textwidth]{pics/pdf-profile-ffabm/q2_100000_pdf_uv_ratio.pdf}}}
    {{\includegraphics[width=0.235\textwidth]{pics/pdf-profile-ffabm/q2_100000_pdf_dv_ratio.pdf}}}
    \caption{The relative strange (top left), gluon (top right), sea quark (middle left), u valence quark (middle right) and d valence quark (bottom) PDF uncertainties at $\mu_\mathrm{f}^2=100000$ GeV$^2$ of the original and profiled \abmp PDF set.}
    \label{fig:pdf-abmp-100000}
\end{figure}

\begin{figure}
    \centering
    {{\includegraphics[width=0.235\textwidth]{pics/pdf-profile-fonll/q2_100_pdf_s_ratio.pdf}}}
    {{\includegraphics[width=0.235\textwidth]{pics/pdf-profile-fonll/q2_100_pdf_g_ratio.pdf}}}\\
    {{\includegraphics[width=0.235\textwidth]{pics/pdf-profile-fonll/q2_100_pdf_Sea_ratio.pdf}}}
    {{\includegraphics[width=0.235\textwidth]{pics/pdf-profile-fonll/q2_100_pdf_uv_ratio.pdf}}}
    {{\includegraphics[width=0.235\textwidth]{pics/pdf-profile-fonll/q2_100_pdf_dv_ratio.pdf}}}
    {{\includegraphics[width=0.235\textwidth]{pics/pdf-profile-fonll/q2_100_pdf_c_ratio.pdf}}}
    \caption{The relative strange (top left), gluon (top right), sea quark (middle left), u valence quark (middle right), d valence quark (bottom left) and charm quark (bottom right) PDF uncertainties at $\mu_\mathrm{f}^2=100$ GeV$^2$ of the original and profiled \nnpdf PDF set.}
    \label{fig:pdf-nnpdf}
\end{figure}

\begin{figure}
    \centering
    {{\includegraphics[width=0.235\textwidth]{pics/pdf-profile-fonll/q2_100000_pdf_s_ratio.pdf}}}
    {{\includegraphics[width=0.235\textwidth]{pics/pdf-profile-fonll/q2_100000_pdf_g_ratio.pdf}}}\\
    {{\includegraphics[width=0.235\textwidth]{pics/pdf-profile-fonll/q2_100000_pdf_Sea_ratio.pdf}}}
    {{\includegraphics[width=0.235\textwidth]{pics/pdf-profile-fonll/q2_100000_pdf_uv_ratio.pdf}}}
    {{\includegraphics[width=0.235\textwidth]{pics/pdf-profile-fonll/q2_100000_pdf_dv_ratio.pdf}}}
    {{\includegraphics[width=0.235\textwidth]{pics/pdf-profile-fonll/q2_100000_pdf_c_ratio.pdf}}}
    \caption{The relative strange (top left), gluon (top right), sea quark (middle left), u valence quark (middle right), d valence quark (bottom left) and charm quark (bottom right) PDF uncertainties at $\mu_\mathrm{f}^2=100000$ GeV$^2$ of the original and profiled \nnpdf PDF set.}
    \label{fig:pdf-nnpdf-100000}
\end{figure}

\section{Discussion and summary}

\begin{acknowledgements}

We would like to thank
John~C.~Collins,
Aleksander~Kusina,
Ted~C.~Rogers,
Ingo~Schienbein,
George~Sterman,
...
for useful discussion.

%  We would like to thank Juan Rojo for a cri\-ti\-cal reading of this
%  paper, and Luca Rottoli for discussions on the description of charm
%  data.  V.~B.\ and F.~G.\ are supported by the European Research
%  Council Starting Grant ``PDF4BSM''.  Additional support was received
%  by A.~G., A.~S.\ and P.~S.\ from the BMBF-JINR cooperation program
%  and the Heisenberg-Landau program.  A.~L.\ is supported by the
%  Polish Ministry under program Mobility Plus, no 1320/MOB/IV/2015/0.
%  M.~B.\ is supported by the by the Marie Sk\l{}odowska Curie grant
%  HiPPiE@LHC.
\end{acknowledgements}


%%%%%%%%%%%%%%%%%%%%%%%%%%%%%%%%%%%%%%%%%
%  %% LyX 2.3.1 created this file.  For more info, see http://www.lyx.org/.
%  %% Do not edit unless you really know what you are doing.
%  \documentclass[twocolumn,english,showpacs,preprintnumbers,amsmath,amssymb,floatfix]{revtex4-1}
%  \usepackage[T1]{fontenc}
%  \usepackage[latin9]{inputenc}
%  \setcounter{secnumdepth}{3}
%  \usepackage{color}
%  \usepackage{babel}
%  \usepackage{graphicx}
%  \usepackage{esint}
%  \usepackage[unicode=true,
%   bookmarks=false,
%   breaklinks=false,pdfborder={0 0 1},backref=section,colorlinks=false]
%   {hyperref}
%  
%  \makeatletter
%  %%%%%%%%%%%%%%%%%%%%%%%%%%%%%% User specified LaTeX commands.
%  \hyphenpenalty=10000
%  
%  \makeatother
%  
%  \begin{document}
%  
\clearpage

\appendix 
\section{Defining $F_{2}^{charm}$ Beyond Lead Order}

\begin{figure}[t]
\centering
\includegraphics[width=0.45\textwidth]{./pics/fred/cancellation}
\caption{\textcolor{blue}{
We may switch to charm; I took from our heavy flavor paper.} 
The LO contributions correspond to the heavy quark ($Q$) initiated $f_{Q}$,
and the SUB to $\tilde{f}_{Q}$. The cancellation (LO-SUB) is quite
precise. If we were to remove LO or SUB, our TOT result would have
anomalous contributions (and correspondingly anomalous $\mu$-dependence)
in the region $\mu\sim m_Q$.}
\end{figure}


\begin{figure}[t]
\centering
\includegraphics[width=0.45\textwidth]{./pics/fred/feyngraph}
\caption{\textcolor{blue}{draft in progress: } 
%
A higher order Feynman graph illustrating the difficulty in defining
an ``inclusive'' $F_2^{charm}$. 
If we have a light quark ($q$) scattering from a vector boson ($V$),
at higher orders we could have a charm--anti-charm loop. 
If we cut the amplitude with cut ``A'' we have charm in the final state
and this must be included in  $F_2^{charm}$. 
If we cut the amplitude with cut ``B'' there is no charm in the final state,
but this process is required to satisfy IR  divergences as governed by the Kinoshita-Lee-Nauenberg (KLN) theorem.
Also note, since this diagram contributes to the beta function, 
this highlights the difficulty of using an $\alpha_S$ and hard scattering $\hat{\sigma}$ with differing $N_{eff}$.
}
\end{figure}



\begin{figure*}
\centering
\includegraphics[clip,width=0.80\textwidth]{./pics/fred/tchannel}
\caption{Gluon NLO $t$-channel processes}
\end{figure*}
%
\begin{figure*}
\centering
\includegraphics[clip,width=0.80\textwidth]{./pics/fred/uchannel}
\caption{Gluon NLO $u$-channel processes}
\end{figure*}


The charged current DIS charm production process involves some interesting
issues. Because two quark masses are involved $\{m_{s},m_{c}\}$,
we can separately examine the mass singularities of the $t$-channel
and $u$-channel separately; this separation is particularly useful
to understand how the FFNS and VFNS divide up the contributions to
the total structure function. Additionally, the DIS charm production
allows us to identify the deficiencies we encounter due to the fact
that a truly ``inclusive'' $F_{2}^{charm}$is not a theoretically
well-defined observable.

x

\textcolor{blue}{{[}FRED: THIS INTRO IS STILL ROUGH{]} Suppose we
attempt to compute the ``inclusive'' $F_{2}^{charm}$ for Charged
Current (CC) charm production at NLO in the VFNS. The obvious LO diagram
is $(sW^{+}\to c).$ What is not so obvious is we also will need ($\bar{c}W^{+}\to\bar{s}$).
This is because the $\bar{c}$ comes from a gluon splitting to $c\bar{c}$,
and the $c$ goes down the beam pipe along with the hadron remnants.
All three $u$-channel terms displayed in Fig.~{*}{*}{*} are required
for the result to be both i) insensitivity to the $\mu$-scale, and
ii) be free of mass singularities at large $Q^{2}$ scales; but this
requires measuring charm quarks in an experimentally unaccessible
region---the hadron remnants. This is why a truly ``inclusive''
$F_{2}^{charm}$ is ill-defined; it is experimentally unobservable.}\footnote{\textcolor{blue}{The proof of factorization for heavy quarks by Collins
cite{*}{*}{*}{*} addressed a fully inclusive $F_{2}$; it specifically
avoided the ill-defined $F_{2}^{charm}$. }}

\textcolor{blue}{What is actually measured experimentally is a differential
charm production process which must include a resolution scale (or
regulator) to make a cut on charm quarks in the beam fragments; this
``non-inclusive'' $F_{2}^{charm}$ (or ``exclusive'') measurement
can be well defined.}

\textcolor{blue}{For this discussion we will focus on the NLO gluon
initiated graphs; there are a parallel set of NLO quark initiated
processes, but the principles are fully illustrated by the gluon processes. }

\textcolor{blue}{Additionally, we note that an ``inclusive'' $F_{2}^{charm}$
in the FFNS is also ill-defined; at higher orders we have $g\to c\bar{c}$
processes which make it impossible to separate out a ``charm only''
contribution from the total $F_{2}$.}\footnote{\textcolor{blue}{See for example. Ref. Smith and van Neerven cite{*}{*}{*}
At ${\cal O}(\alpha_{S}^{3})$ we can have internal $g\to c\bar{c}$
processes which make a $F_{2}^{charm}$ definition ambiguous. This
issues is particularly problematic in beta-function which sums over
internal quark loops and determines the running of $\alpha_{S}$. }}

\subsection{t-channel at NLO}

The t-channel contributions at NLO are straightforward. We start with
a leading-order (LO) $sW^{+}\to c$ process. We then add the next-to-leading-order
(NLO) $gW^{+}\to c\bar{s}$ diagram; this exchanges an $s$ quark
in the t-channel, and thus will have a $\ln(m_{s}^{2}/Q^{2})$ divergence
for large $Q$. This is resolved by the subtraction (SUB) term $f_{g}\otimes{\cal P}_{g\to s}\otimes\sigma_{sW^{+}\to c}$
where the ${\cal P}_{g\to s}$ represents a perturbative splitting
of $g\to s$; the SUB term is proportional to ${\cal P}_{g\to s}\sim\frac{\alpha_{S}}{2\pi}\:P_{g\to s}^{(1)}\:\ln(m_{s}^{2}/Q^{2})$,
and will cancel the double counting between the LO and NLO graphs
in the limit where the exchanged $s$ quark becomes collinear.\footnote{Here we use ${\cal P}_{g\to s}$ to represent the perturbative splitting
contribution which at NLO is given by $\frac{\alpha_{S}}{2\pi}\:P_{g\to s}^{(1)}\:\ln(m_{s}^{2}/Q^{2})$,
where $P_{g\to s}^{(1)}$ is the usual DGLAP splitting kernel.} The logarithmic divergence (mass singularity) will cancel between
the NLO and SUB terms as $Q^{2}\to\infty$, resulting in a finite
result for the NLO t-channel contribution. 

\subsection{u-channel at NLO}

The u-channel at NLO is more subtle. We definitely need the NLO $gW^{+}\to c\bar{s}$
diagram with a $\bar{c}$ quark exchanged in the u-channel, and thus
will have a $\ln(m_{c}^{2}/Q^{2})$ divergence for large $Q$. This
is resolved by the subtraction term $f_{g}\otimes{\cal P}_{g\to\bar{c}}\otimes\sigma_{\bar{c}W^{+}\to\bar{s}}$
where the ${\cal P}_{g\to\bar{c}}$ represents a perturbative splitting
of $g\to\bar{c}$ and will cancel the double counting between the
LO and NLO graphs in the limit where the exchanged $\bar{c}$ quark
becomes collinear. Here, the SUB term is proportional to ${\cal P}_{g\to\bar{c}}\sim\frac{\alpha_{S}}{2\pi}\:P_{g\to\bar{c}}^{(1)}\:\ln(m_{c}^{2}/Q^{2})$.
The logarithmic divergence will cancel between the NLO and SUB terms
as $Q^{2}\to\infty$, resulting in a finite result for the NLO u-channel
contribution. 

\subsection{Why do we need the LO ($\bar{c}W^{+}\to\bar{s}$)?}

What is not so obvious is that we need the LO $u$-channel process
$\bar{c}W^{+}\to\bar{s}$.

Recall that it is essential we include the subtraction SUB term $f_{g}\otimes{\cal P}_{g\to\bar{c}}\otimes\sigma_{\bar{c}W^{+}\to\bar{s}}$
so that we get a finite answer at large energies $Q^{2}\to\infty$. 

At energy scales $Q\sim m_{c}$, the LO and SUB terms remove the double
counting between the LO and NLO processes. This is most apparent when
you plot the individual terms versus the $Q$ scale (or more properly,
it is the $\mu$ scale). {[}See Figure{]} In the region of $Q\sim m_{c}$,
the charm PDF $f_{c}$ (and hence, the LO contribution) rises very
quickly as it is driven by the very large gluon, and coupled with
a large $\alpha_{S}(m_{c})$. The SUB subtraction also rises quickly
as this is driven by the logarithmic term $\ln(m_{c}^{2}/Q^{2})$.
The difference LO-SUB is the physical contribution to the total (TOT=LO+NLO-SUB),
and it is this combination which is smooth across the ``turn on''
of the charm PDF. We now see that if we neglect the LO ($\bar{c}W^{+}\to\bar{s}$)
we loose the cancellation between LO and SUB in the $Q\sim m_{c}$
and our structure function (or cross section) would have an anomalous
shift at the location where we arbitrarily turn on the charm PDF. 

So to recap, the combination of the LO and SUB terms ensure a minimal
$\mu$-variation at low $\mu$ scales, and the combination of SUB
and NLO ensure the mass singularities are canceled at high $\mu$
scales. 

\subsection{FFNS: u-channel for $N_{F}=3$}

Let us clarify the case where we work in a FFNS with 3 flavors $\{u,d,s\}$
but no charm PDF. In this case there is no LO ($\bar{c}W^{+}\to\bar{s}$)
process as $f_{c}=0$, and there is no u-channel subtraction $f_{g}\otimes{\cal P}_{g\to\bar{c}}\otimes\sigma_{\bar{c}W^{+}\to\bar{s}}$.
This is all perfectly consistent. However, the NLO u-channel process
($gW^{+}\to c\bar{s}$) will have a potentially divergent $\ln(m_{c}^{2}/Q^{2})$
contribution from the exchanged charm quark; this is fine so long
as we don't go to large $Q$. If we do want large $Q$, then we will
need to resum the $\ln(m_{c}^{2}/Q^{2})$ logs using the charm PDF. 

We expect this FFNS to diverge from the VFNS result by contributions
proportional to $\sim\frac{\alpha_{S}}{2\pi}\:\ln(m_{c}^{2}/Q^{2})$.

\subsection{The bottom line: }

A truly ``inclusive'' $F_{2}^{charm}$ is ill-defined. Instead,
we necessarily must an ``experimentally'' defined $F_{2}^{charm}$
where we specify conditions so that the final state charm is isolated
from the hadron remnants. 

We can talk about a fully inclusive $F_{2}$ where we include all
flavors; this was the subject of Collins' proof. 

If we compute a ``pseudo-inclusive'' $F_{2}^{charm}$ in the Variable
Flavor Number Scheme, we do need to include the LO ($\bar{c}W^{+}\to\bar{s}$)
and the associated SUB ($gW^{+}\to c\bar{s}$).

We can compute in the Fixed Flavor Number Scheme, but in the large
energy limit, we encounter $\ln(m_{c}^{2}/Q^{2})$ divergences. In
practice, our $Q$ scales are not large enough to generate infinities,
but they are large enough where we see the resummed logs included
in the VFNS charm PDF become important. Regardless, the FFNS is also
unable to define a truly ``inclusive'' $F_{2}^{charm}$.

%\end{document}

%%%%%%%%%%%%%%%%%%%%%%%%%%%%%%%%%%%%%%%%%


\bibliographystyle{spphys}
\bibliography{c_cpaper}

\end{document}
